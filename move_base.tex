\documentclass{llncs}

\usepackage[T1]{fontenc} \usepackage[utf8]{inputenc} \usepackage[german]{babel}

\usepackage{cite} %Vereinfachung der Referenzen: [1], [2], [3] => [1-3]
\usepackage{makeidx}         % allows index generation

\usepackage{graphicx} \usepackage[cmex10]{amsmath} %mathamatische Formeln %\usepackage{stfloat} %Bilder über zwei Spalten &\usepackage{url}
\makeindex             % used for the subject index

\begin{document} \title{ROS - Move Base}

\author{Nicolas Limpert \and Christian Schnieder} \institute{Fachhochschule Aachen - University of Applied Sciences \and Robotik WS 2014 / 2015 }

\maketitle
\tableofcontents


\begin{abstract} Das ROS-Package move\_base stellt  die Möglichkeit dar mit gegebener Karte und Anfangsposition ein örtliches Ziel mit einem Roboter zu erreichen, dessen TODO \end{abstract}

\section{Einleitung}
Move Base ist ein ROS-Paket das dazu verwendet wird einem Roboter ein örtliches Ziel zu vermitteln und dieses Ziel in Kombination von mehreren ROS-Nodes (globalen- und lokalen Planer, globale und lokale costmap, etc.) zu versuchen, zu erreichen.\\
Das Ziel wird der Move Base in Form einer Action vermittelt, mit der Idee aus einer ROS-Message nach Möglichkeit eine Reihe von Fahrbefehlen unter Berücksichtigung von Kollisionsvermeidung, optimaler Pfadplanung (in Abhängigkeit von lokalem und globalem Planner) ausuzführen.

\section{Komponenten}
Move\_base besteht aus folgenden Komponenten:
\begin{description}
\item[Globaler Planer]
Dies kann ein beliebiger Planer sein, solange er das Interface nav\_core::BaseGlobalPlanner erfüllt. Er ist für die Pfadplanung innerhalb der Karte zuständig, führt also beispielsweise eine A* - Suche durch, um von einem gegebenen Anfangspunkt den gewünschten Endpunkt zu erreichen.
\item[Lokaler Planer]
Dies kann ein beliebiger Planer sein, solange er das Interface nav\_core::BaseLocalPlanner erfüllt. Aufgabe des lokalen Planers ist das Erreichen des nächsten Punkts den der globale Planer gegeben hat, bzw. das ausgeben von Fahrbefehlen zum erreichen dieses nächsten Punkts.
\item[Globale Costmap]
Dies kann ein beliebiger Planer sein, solange er das Interface nav\_core::BaseLocalPlanner erfüllt
\item[Lokale Costmap]
Dies kann ein beliebiger Planer sein, solange er das Interface nav\_core::BaseLocalPlanner erfüllt
\end{description}

\section{Verwandte Arbeiten/Stand der Technik}

asiuhfhfsauhafds
\section{Technischer Hintergrund} \section{Eigene Arbeit} \section{Evaluation} \section{Zusammenfassung/Ausblick}

\bibliographystyle{IEEEtran} \bibliography{paper} %paper ist Literaturverzeichnis

\end{document}
